\usepackage[a4paper,top=2cm,bottom=2.5cm,left=1.5cm,right=1.5cm,marginparwidth=1.75cm]{geometry}

% Language setting
\usepackage[ngerman]{babel}
% deutsche Umlaute
\usepackage[utf8]{inputenc}
\usepackage[T1]{fontenc}

%Schriftart
\renewcommand{\familydefault}{\sfdefault}

\usepackage[activate]{microtype}
%Abb. anstatt Abbildung
\renewcaptionname{ngerman}{\figurename}{Abb.}
\usepackage[dcucite]{harvard}
\renewcommand{\harvardand}{und}

%damit nur fref verwendet werden kann umm einen verweis auf eine Abbildung einzufügen
\newcommand{\fref}[1]{\figurename\ \ref{#1}}

%Schriftgrösse Abb. und Tabelle
\usepackage{caption}
\captionsetup[figure]{font=scriptsize}
\captionsetup[table]{font=scriptsize}

%um Tabellen farblich zu gestalten
\usepackage[table]{xcolor}

\usepackage{float}
% Useful packages
\usepackage{amsmath}
\usepackage{graphicx}
\usepackage[colorlinks=true, allcolors=black]{hyperref}
\setlength{\parindent}{0pt}
\usepackage{setspace}
%Keine Silbentrennung
% \usepackage[none]{hyphenat}
\tolerance=1
\emergencystretch=\maxdimen 
\hbadness=10000
% to use a for loop
\usepackage{pgffor}

%Tabellen mit Zeilenumbruch
\usepackage{tabulary} 
\usepackage{tabularx} 

\usepackage{scrlayer-fancyhdr}
% Turn on the style
\pagestyle{fancy}
% Clear the footer
\fancyfoot{}
% Set the right side of the footer to be the page number
\fancyfoot[R]{Seite | \thepage}