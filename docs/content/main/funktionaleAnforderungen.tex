\begin{table}[H]
  \centering
  \settowidth\tymin{\textbf{Prio}}
  \setlength\extrarowheight{2pt}
  \begin{tabulary}{1.0\textwidth}{|m{16cm}|L|}
    \hline
    \textbf{Datenhaltung}&\textbf{Prio}\\
    \hline
      Die Daten der Applikation müssen in einer Datenbank gespeichert werden & 1\\
    \hline
  \end{tabulary}
  \caption{FA-Datenhaltung}
  \label{faDatenhaltung}
\end{table}

\begin{table}[H]
  \centering
  \settowidth\tymin{\textbf{Prio}}
  \setlength\extrarowheight{2pt}
  \begin{tabulary}{1.0\textwidth}{|m{16cm}|L|}
    \hline
    \textbf{Mietverträge verwalten}&\textbf{Prio}\\
    \hline
    In der Applikation müssen die Mietverträge mit folgenden Angaben erstellt werden können:
    \begin{itemize}
      \item Welcher Mieter:in, bei Wohngemeinschaften/Ehepaaren muss es möglich sein mehrere Mieter pro Mietvertrag zu erfassen
      \item Welche Liegenschaft / Objekt
      \item Der Mietbeginn und gegebenenfalls Mietende
      \item Der Monatlicher Mietzins und die monatlichen Nebenkosten
      \item Die Art der Nebenkosten (Pauschal bzw. à-Konto)
      \item Eine Auflistung der Kostenarten, welche über die Nebenkosten abgerechnet werden (Heiz-, Warmwasseraufbereitungs-, Wasser-, Hauswart-, Treppenreinigungs-, Gartenarbeits-, Strom-, Lift-, Kabelfernsehkosten sowie Abwasser- Kehrichtgebühren)
      \item Ob ein Mietdepot eingerichtet wurde und wenn ja der Depotbetrag in CHF
    \end{itemize}  & 1\\ 
    \hline
  \end{tabulary}
  \caption{FA-Mietverträge}
  \label{faMietverträge}
\end{table}

\begin{table}[H]
  \centering
  \settowidth\tymin{\textbf{Prio}}
  \setlength\extrarowheight{2pt}
  \begin{tabulary}{1.0\textwidth}{|m{16cm}|L|}
    \hline
    \textbf{Liegenschaft erfassen}&\textbf{Prio}\\
    \hline
    Die Liegenschaft muss mit folgenden Informationen in der Applikation erfasst werden können:
    \begin{itemize}
      \item Eine Identifikationsnummer der Liegenschaft
      \item Eine Bezeichnung der Liegenschaft
      \item Die Liegenschaftsadresse
      \item Wie viele Objekte pro Kategorie in der Liegenschaft vorhanden sind. Objektkategorie pro Liegenschaft sind: Wohnung, Raum, Garage, Parkplatz
      \item Abgeschlossene Versicherungen für die Liegenschaft
      \item Die zuständige Hauswartungsperson
    \end{itemize} & 1\\ 
    \hline
  \end{tabulary}
  \caption{FA-Liegenschaft erfassen}
  \label{faLiegenschaftErfassen}
\end{table}

\begin{table}[H]
  \centering
  \settowidth\tymin{\textbf{Prio}}
  \setlength\extrarowheight{2pt}
  \begin{tabulary}{1.0\textwidth}{|m{16cm}|L|}
    \hline
    \textbf{Objekt erfassen}&\textbf{Prio}\\
    \hline
    Ein Objekt muss mit folgenden Informationen in der Applikation erfasst werden können:
    \begin{itemize}
      \item Eine Identifikationsnummer des Objekts
      \item Eine Bezeichnung des Objekts
      \item Die Lage des Objekts. Z.B. 1. OG links
      \item Die Anzahl der Zimmer beim Objekttyp Wohnungen
      \item Die Fläche des Objekts
      \item Eine Beschreibung als Freitextfeld
      \item Was für Geräte in dem Objekt vorhanden sind. Z.B. Kühlschrank, Geschirrwaschmaschine, etc.
      \item Die Anzahl aller Schlüssel nach Typ, die zu diesem Objekt vorhanden sind 
    \end{itemize} & 1\\ 
    \hline
  \end{tabulary}
  \caption{FA-Objekt erfassen}
  \label{faObjektErfassen}
\end{table}

\begin{table}[H]
  \centering
  \settowidth\tymin{\textbf{Prio}}
  \setlength\extrarowheight{2pt}
  \begin{tabulary}{1.0\textwidth}{|m{16cm}|L|}
    \hline
    \textbf{Mieter:in erfassen}&\textbf{Prio}\\
    \hline
    Die Mieter müssen mit folgenden Informationen in der Applikation erfasst werden können:
    \begin{itemize}
      \item Der Name und Vorname
      \item Das Geburtsdatum
      \item Der Heimatort
      \item Der Zivilstand
      \item Die vorherige Wohnadresse
      \item Eine Telefonnummer
      \item Eine E-Mail-Adresse
      \item Eine Bankverbindung
    \end{itemize} & 1\\
    \hline
  \end{tabulary}
  \caption{FA-Mieter:in erfassen}
  \label{faMieterinerfassen}
\end{table}

\begin{table}[H]
  \centering
  \settowidth\tymin{\textbf{Prio}}
  \setlength\extrarowheight{2pt}
  \begin{tabulary}{1.0\textwidth}{|m{16cm}|L|}
    \hline
    \textbf{Buchhaltung}&\textbf{Prio}\\
    \hline
      Es muss eine einfache Buchhaltung mit Ein- und Ausgaben geführt werden, wo die Einzahlungen der Mieter:innen (Mietzinse) und die Ausgaben für den Unterhalt und die Reparaturen verbucht werden können.& 1\\
    \hline
  \end{tabulary}
  \caption{FA-Buchhaltung}
  \label{faBuchhaltung}
\end{table}

\begin{table}[H]
  \centering
  \settowidth\tymin{\textbf{Prio}}
  \setlength\extrarowheight{2pt}
  \begin{tabulary}{1.0\textwidth}{|m{16cm}|L|}
    \hline
    \textbf{Kreditoren verwalten}&\textbf{Prio}\\
    \hline
    Es müssen Kreditoren mit folgenden Informationen verwaltet werden können:
    \begin{itemize}
      \item Aktiv oder gesperrt
      \item Eine Adresse
      \item Der Name und Vorname
      \item Die Mehrwertsteuernummer
      \item Eine Telefonnummer und falls vorhanden eine Faxnummer
      \item Eine E-Mail-Adresse
      \item Eine Kontaktperson mit Namen, Vorname, Telefonnummer, E-Mail-Adresse
    \end{itemize} & 1\\
    \hline
  \end{tabulary}
  \caption{FA-Kreditoren verwalten}
  \label{faKreditorenVerwalten}
\end{table}

\begin{table}[H]
  \centering
  \settowidth\tymin{\textbf{Prio}}
  \setlength\extrarowheight{2pt}
  \begin{tabulary}{1.0\textwidth}{|m{16cm}|L|}
    \hline
    \textbf{Rechnungen erstellen}&\textbf{Prio}\\
    \hline
    Es muss eine Rechnung mit folgenden Informationen erstellt werden können: (für den Prototyp ist keine Ausgabe als PDF oder sonstige Datei notwendig.)
    \begin{itemize}
      \item Das Konto
      \item Der Kreditor
      \item Das Rechnungsdatum
      \item Das Fälligkeitsdatum der Rechnung
      \item Der Rechnungszweck
      \item Die betreffende Liegenschaft oder das betreffende Objekt
      \item Der Rechnungsbetrag mit evtl. Rabatt/Skonto, der Betrag Netto
      \item Die Kategorie als Auswahl Nebenkosten oder Allgemein
    \end{itemize} & 1\\
    \hline
  \end{tabulary}
  \caption{FA-Kreditoren verwalten}
  \label{faRechnungenErstellen}
\end{table}

\begin{table}[H]
  \centering
  \settowidth\tymin{\textbf{Prio}}
  \setlength\extrarowheight{2pt}
  \begin{tabulary}{1.0\textwidth}{|m{16cm}|L|}
    \hline
    \textbf{Mietzinskontrolle}&\textbf{Prio}\\
    \hline
    Es muss überprüft werden können, ob die Miete für ein bestimmtes Objekt schon bezahlt wurde und wenn nicht, muss eine Mahnung für den Mieter generiert werden können. (Mietzinsinkasso) & 1\\
    \hline
  \end{tabulary}
  \caption{FA-Mietzinskontrolle}
  \label{faMietzinskontrolle}
\end{table}

\begin{table}[H]
  \centering
  \settowidth\tymin{\textbf{Prio}}
  \setlength\extrarowheight{2pt}
  \begin{tabulary}{1.0\textwidth}{|m{16cm}|L|}
    \hline
    \textbf{Daten verändern}&\textbf{Prio}\\
    \hline
    Es müssen die Daten der Mieter:innen, die Daten je Liegenschaft und die Daten je Objekt verändert werden können. & 1\\
    \hline
  \end{tabulary}
  \caption{FA-Daten verändern}
  \label{faDatenVeraendern}
\end{table}

\begin{table}[H]
  \centering
  \settowidth\tymin{\textbf{Prio}}
  \setlength\extrarowheight{2pt}
  \begin{tabulary}{1.0\textwidth}{|m{16cm}|L|}
    \hline
    \textbf{Login in der Applikation}&\textbf{Prio}\\
    \hline
    Jede Mitarbeiter:in muss sich mit seinem Persönlichen Login in der Applikation einloggen können & 1\\
    \hline
  \end{tabulary}
  \caption{FA-Login in der Applikation}
  \label{faLogininderApplikatio}
\end{table}