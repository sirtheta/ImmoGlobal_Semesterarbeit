\section{Konzept}

\subsection{Kontextdiagramm}
\begin{figure}[H]
  \begin{center}
    \includegraphics[width=0.99\linewidth]{content/diagrams/out/contextdiagram/context.png}
    \caption{Kontextdiagramm}
  \end{center}
  \label{contextdiag}
\end{figure}

\subsection{Geschäftsprozesse}
\begin{table}[H]
  \newcolumntype{a}{>{\columncolor[HTML]{4473C5}}L}
  \centering
  \settowidth\tymin{\textbf{Kurzbeschreibung}}
  \setlength\extrarowheight{2pt}
  \begin{tabulary}{1.0\textwidth}{|a|m{12cm}|}
    \hline
    \textbf{Name}& Verwaltung von Mietobjekten \\
    \hline 
    \textbf{Kurzbeschreibung} & Alle Geschäftsanwendungsfälle die nötig sid um Mietobjekte korrekt verwalten zu können und im Aufgabenbereich von ImmoGlobal liegen\\
    \hline
    \textbf{Geschäftsanwendungsfälle} & 
    \begin{itemize}
      \item Verwaltung von Objekten
      \item Verwaltung von Liegenschaften
      \item Übernahme- und Übergabeprotokoll der Mietobjekte
      \item Verwaltung der Mieter:innen
      \item Erstellen / Verwalten der Mietverträge
      \item Erfassen von Ein- und Ausgaben  (Mietzins, Nebenkosten, Gebühren für Unterhalt der Liegenschaft, etc.)
      \item Mietzins- und Nebenkostenkontrolle
      \item Rechnung erstellen
    \end{itemize}\\
    \hline
    \textbf{Verantwortlichkeit} & Geschäftsleiter\\
    \hline
    \textbf{Beteiligte} & 
    \begin{itemize}
      \item Mitarbeiter der Administration
      \item Hauswartungspersonen
      \item Mieter
    \end{itemize}\\
    \hline
  \end{tabulary}
  \caption{Geschäftsprozesse}
\end{table}


\subsection{Geschäftsanwendungsfälle}
\begin{table}[H]
  \newcolumntype{a}{>{\columncolor[HTML]{4473C5}}L}
  \centering
  \settowidth\tymin{\textbf{Kurzbeschreibung}}
  \setlength\extrarowheight{2pt}
  \begin{tabulary}{1.0\textwidth}{|a|m{12cm}|}
    \hline
    \textbf{Name}& Verwaltung von Objekten\\
    \hline 
    \textbf{Kurzbeschreibung} & Eine bestehendes Objekt muss editiert werden oder eine neues Objekt soll hinzugefügt werden \\
    \hline
    \textbf{Akteure} & Mitarbeiter in der Liegenschaftsverwaltung\\
    \hline
    \textbf{Auslöser} & Für die erfasste Liegenschaft wurden noch keine Objekte hinzugefügt\newline 
    Ein bestehendes Objekt muss ergänzt/korrigiert werden\\
    \hline
    \textbf{Ergebnis} & Das veränderte oder neu erstellte Objekt kann zu einer Liegenschaft hinzugefügt werden\\
    \hline
    \textbf{Eingehende Daten} & Informationen zum erfassenden Objekt\\
    \hline
    \textbf{Vorbedingungen} & Keine\\
    \hline
    \textbf{Nachbedingungen} & Keine\\
    \hline
    \textbf{Ablauf} & Der Benutzer trägt alle Muss-Daten für das neue Objekt ein und speichert es ab. \\
    \hline
  \end{tabulary}
  \caption{GA-Verwaltung von Objekten}
\end{table}

\begin{table}[H]
  \newcolumntype{a}{>{\columncolor[HTML]{4473C5}}L}
  \centering
  \settowidth\tymin{\textbf{Kurzbeschreibung}}
  \setlength\extrarowheight{2pt}
  \begin{tabulary}{1.0\textwidth}{|a|m{12cm}|}
    \hline
    \textbf{Name}& Verwaltung von Liegenschaften\\
    \hline 
    \textbf{Kurzbeschreibung} & Eine bestehende Liegenschaft muss editiert werden oder eine neue Liegenschaft soll hinzugefügt werden\\
    \hline
    \textbf{Akteure} & Mitarbeiter in der Liegenschaftsverwaltung\\
    \hline
    \textbf{Auslöser} & Die Verwaltung wird mit dem verwalten einer neuen Liegenschaft beauftragt \newline
    Bei einer Liegenschaft müssen Änderungen vorgenommen werden\\
    \hline
    \textbf{Ergebnis} & Die neuen Daten zu der Liegenschaft werden abgespeichert und in der Übersicht dargestellt\\
    \hline
    \textbf{Eingehende Daten} & Liegenschaft mit allen dazu gehörigen Informationen\\
    \hline
    \textbf{Vorbedingungen} & Die Daten zur Liegenschaft müssen bekannt sein\\
    \hline
    \textbf{Nachbedingungen} & Keine\\
    \hline
    \textbf{Ablauf} & Der Eingeloggte Mitarbeiter trägt alle Muss-Daten für die neue Liegenschaft ein und speichert diese. Der Mitarbeiter kann den Objekttyp für die Liegenschaft auswählen und die Informationen pro Objekt eintragen. Wenn der Objekttyp noch nicht erfasst wurde, muss der Objekttyp noch erstellt werden\newline
    Beim editieren einer Liegenschaft werden die Vorhandenen Bearbeitet. Falls ein neuer Objekttyp hinzugefügt werden muss der noch nicht in der Datenbank vorhanden ist, muss dieser wie bei einer neuen Liegenschaft, hinzugefügt werden\\
    \hline
  \end{tabulary}
  \caption{GA-Verwaltung von Liegenschaften}
\end{table}

\begin{table}[H]
  \newcolumntype{a}{>{\columncolor[HTML]{4473C5}}L}
  \centering
  \settowidth\tymin{\textbf{Kurzbeschreibung}}
  \setlength\extrarowheight{2pt}
  \begin{tabulary}{1.0\textwidth}{|a|m{12cm}|}
    \hline
    \textbf{Name}& Übernahme- und Übergabeprotokoll der Mietobjekte\\
    \hline 
    \textbf{Kurzbeschreibung} &\\
    \hline
    \textbf{Akteure} & Mitarbeiter in der Liegenschaftsverwaltung, Hauswartungsperson, Mieter\\
    \hline
    \textbf{Auslöser} & Beginn- oder Ende eines Mietverhältnisses\\
    \hline
    \textbf{Ergebnis} & Ein Übernahme- und Übergabeprotokoll für das Mietobjekt\\
    \hline
    \textbf{Eingehende Daten} &  
      $\bullet$ Kündigung \newline
      $\bullet$ Mietvertrag \newline
      $\bullet$ Bei Übergabe zurück an die Verwaltung das Übernahmeprotokoll\\
    \hline
    \textbf{Vorbedingungen} & Keine\\
    \hline
    \textbf{Nachbedingungen} & Keine\\
    \hline
    \textbf{Ablauf} & Für die Übernahme und die Übergabe wird von dem zuständigen Mitarbeiter das Protokoll ausgefüllt und von den Mietparteien auf Vollständigkeit geprüft und unterschrieben. Der Prozess findet komplett Digital statt. Unterschrieben wird das Protokoll auf einem unterschriftsfähigen Tablet. Anschliessend wird daraus ein PDF (PDF/A) generiert und dem Mieter wahlweise per E-Mail oder ausgedruckt per Post zugesandt\\
    \hline
  \end{tabulary}
  \caption{GA-Übernahme- und Übergabeprotokoll der Mietobjekte}
\end{table}

\begin{table}[H]
  \newcolumntype{a}{>{\columncolor[HTML]{4473C5}}L}
  \centering
  \settowidth\tymin{\textbf{Kurzbeschreibung}}
  \setlength\extrarowheight{2pt}
  \begin{tabulary}{1.0\textwidth}{|a|m{12cm}|}
    \hline
    \textbf{Name}& Verwaltung der Mieter:innen\\
    \hline 
    \textbf{Kurzbeschreibung} & Neue Mieter:innen werden in der Applikation hinzugefügt oder bestehende Mieter:innen bearbeitet\\
    \hline
    \textbf{Akteure} & Mitarbeiter in der Liegenschaftsverwaltung, Mieter\\
    \hline
    \textbf{Auslöser} & 
      $\bullet$ Vermietung eines Objektes an einen neuen Mieter:in \newline
      $\bullet$ Namensänderungen \newline
      $\bullet$ Zivilstandänderungen \newline
      $\bullet$ Geburten\\
    \hline
    \textbf{Ergebnis} &
      $\bullet$ Ein neuer Mieter:in wurde in der Applikation erfasst \newline
      $\bullet$ Änderung an bestehendem Mieter:in wurden durchgeführt\\
    \hline
    \textbf{Eingehende Daten} & Informationen über den Mieter:in\\
    \hline
    \textbf{Vorbedingungen} & Die Informationen über den Mieter:in müssen bekannt sein\\
    \hline
    \textbf{Nachbedingungen} & Keine\\
    \hline
    \textbf{Ablauf} & Der Benutzer der Applikation erstellt einen neuen Mieter:in. Er gibt alle Informationen ein und Speichert die Daten ab. Bei einer Änderung eines bestehenden Mieter:in, wird dieser in der Applikation gesucht und ausgewählt, editiert und abgespeichert\\
    \hline
  \end{tabulary}
  \caption{GA-Verwaltung der Mieter:innen}
\end{table}

\begin{table}[H]
  \newcolumntype{a}{>{\columncolor[HTML]{4473C5}}L}
  \centering
  \settowidth\tymin{\textbf{Kurzbeschreibung}}
  \setlength\extrarowheight{2pt}
  \begin{tabulary}{1.0\textwidth}{|a|m{12cm}|}
    \hline
    \textbf{Name}& Erstellen / Verwalten der Mietverträge\\
    \hline 
    \textbf{Kurzbeschreibung} & \\
    \hline
    \textbf{Akteure} & Mitarbeiter in der Liegenschaftsverwaltung, Mieter:in\\
    \hline
    \textbf{Auslöser} & Bewerbung eines neuen Mieters wurde akzeptiert\\
    \hline
    \textbf{Ergebnis} & Gültiger (unterschriebener) Mietvertrag\\
    \hline
    \textbf{Eingehende Daten} & 
      $\bullet$ Bewerbung mit allen Daten zum zukünftigen Mieter \newline
      $\bullet$ Betreibungsregisterauszug\\
    \hline
    \textbf{Vorbedingungen} & 
      $\bullet$ Bewerbung des Mieters wurde akzeptiert \newline
      $\bullet$ Bonität des Mieters wurde überprüft\\
    \hline
    \textbf{Nachbedingungen} & Keine\\
    \hline
    \textbf{Ablauf} & Der Benutzer der Applikation erstellt im betreffenden Mietobjekt einen neuen Mietvertrag. Den Mieter wählt er aus der Liste der vorhandenen Mieter aus. Falls der neue Mieter noch nicht erfasst wurde, kann er diesen erstellen. Anschliessend wird der neue Mietvertrag ausgedruckt und kann von beiden Parteien unterschrieben werden. Nachdem der Mietvertrag unterschrieben ist, wird dieser als PDF in der Applikation abgelegt und als Gültig markiert. Bei einer Kündigung des Mietverhältnisses wird die Kündigung des Mietvertrags in die Applikation hochgeladen und der der Status auf gekündigt geändert. Sobald dann das ende des Mietverhältnisses erreicht ist, wird der Status auf beendet gesetzt.\\
    \hline
  \end{tabulary}
  \caption{GA-Erstellen / Verwalten der Mietverträge}
\end{table}

\begin{table}[H]
  \newcolumntype{a}{>{\columncolor[HTML]{4473C5}}L}
  \centering
  \settowidth\tymin{\textbf{Kurzbeschreibung}}
  \setlength\extrarowheight{2pt}
  \begin{tabulary}{1.0\textwidth}{|a|m{12cm}|}
    \hline
    \textbf{Name}& Erfassen von Ein- und Ausgaben\\
    \hline 
    \textbf{Kurzbeschreibung} & Alle Einnahmen und alle Ausgaben werden erfasst\\
    \hline
    \textbf{Akteure} & Mitarbeiter in der Liegenschaftsverwaltung\\
    \hline
    \textbf{Auslöser} & Zahlungs-Eingang / Ausgang\\
    \hline
    \textbf{Ergebnis} & Neuer Zahlungs-Ein/Ausgang ist im System erfasst\\
    \hline
    \textbf{Eingehende Daten} &
      $\bullet$ Rechnungen \newline
      $\bullet$ Abrechnungen\\
    \hline
    \textbf{Vorbedingungen} & Keine\\
    \hline
    \textbf{Nachbedingungen} & Keine\\
    \hline
    \textbf{Ablauf} & Der Benutzer trägt alle Zahlungseingänge im System ein. Wenn eine Zahlung anhand einer Rechnung getätigt werden muss, bezahlt er diese und trägt den Rechnungsbetrag ebenfalls im System ein\\
    \hline
  \end{tabulary}
  \caption{GA-Kreditor erfassen}
\end{table}

\begin{table}[H]
  \newcolumntype{a}{>{\columncolor[HTML]{4473C5}}L}
  \centering
  \settowidth\tymin{\textbf{Kurzbeschreibung}}
  \setlength\extrarowheight{2pt}
  \begin{tabulary}{1.0\textwidth}{|a|m{12cm}|}
    \hline
    \textbf{Name}&Mietzins- und Nebenkostenkontrolle\\
    \hline 
    \textbf{Kurzbeschreibung} & Der Eingang der Mietzinse und der Nebenkostenzahlungen wird kontrolliert und bei fehlenden Eingängen wird eine Mahnung erstellt\\
    \hline
    \textbf{Akteure} & Mitarbeiter in der Liegenschaftsverwaltung\\
    \hline
    \textbf{Auslöser} & Mietzins- und Nebenkostenzahlungsüberprüfung\\
    \hline
    \textbf{Ergebnis} & Bei negativer Prüfung eine Mahnung\newline 
    Bei positiver Prüfung nichts\\
    \hline
    \textbf{Eingehende Daten} & Mietvertrag\\
    \hline
    \textbf{Vorbedingungen} & Es müssen gültige Mietverträge im System erfasst sein \\
    \hline
    \textbf{Nachbedingungen} & Keine \\
    \hline
    \textbf{Ablauf} & Der Benutzer überprüft ob für ein entsprechendes Objekte/Liegenschaft die Miete und die Nebenkosten bezahlt wurde. Wenn die Miete oder die Nebenkosten noch nicht bezahlt wurde, kann er direkt eine vorgefertigte Mahnung generieren und ausdrucken.\\
    \hline
  \end{tabulary}
  \caption{GA-Mietzins- und Nebenkostenkontrolle}
\end{table}


\begin{table}[H]
  \newcolumntype{a}{>{\columncolor[HTML]{4473C5}}L}
  \centering
  \settowidth\tymin{\textbf{Kurzbeschreibung}}
  \setlength\extrarowheight{2pt}
  \begin{tabulary}{1.0\textwidth}{|a|m{12cm}|}
    \hline
    \textbf{Name}&Rechnung erstellen\\
    \hline 
    \textbf{Kurzbeschreibung} & Eine neue Rechnung erstellen\\
    \hline
    \textbf{Akteure} & Mitarbeiter in der Liegenschaftsverwaltung\\
    \hline
    \textbf{Auslöser} & Es wurde eine Dienstleistung, die die Liegenschaftsverwaltung getätigt hat, bezogen \newline 
    Für eine Liegenschaft oder ein Objekt wurde die Nebenkostenabrechnung erstellt\\
    \hline
    \textbf{Ergebnis} & Eine Rechnung\\
    \hline
    \textbf{Eingehende Daten} & Informationen warum die Rechnung erstellt werden muss \\
    \hline
    \textbf{Vorbedingungen} & Keine\\
    \hline
    \textbf{Nachbedingungen} & Keine\\
    \hline
    \textbf{Ablauf} & Der Benutzer erstellt eine neue Rechnung und gibt alle nötigen Informationen für die Rechnung ein.\\
    \hline
  \end{tabulary}
  \caption{GA-Rechnung erstellen}
\end{table}

\subsection{Use-Case-Beschreibungen}
\begin{figure}[H]
  \begin{center}
    \includegraphics[width=0.6\linewidth]{content/diagrams/out/usecase/login/Login.png}
    \caption{Usecase Login}
  \end{center}
  \label{login}
\end{figure}

\begin{figure}[H]
  \begin{center}
    \includegraphics[width=0.5\linewidth]{content/diagrams/out/usecase/liegenschaftErfassen/LiegenschaftErfassen.png}
    \caption{Usecase Liegenschaft erfassen}
  \end{center}
  \label{Liegenschaft}
\end{figure}

\begin{figure}[H]
  \begin{center}
    \includegraphics[width=0.8\linewidth]{content/diagrams/out/usecase/objektErfassen/ObjektErfassen.png}
    \caption{Usecase Objekt Erfassen}
  \end{center}
  \label{objekt}
\end{figure}

\begin{figure}[H]
  \begin{center}
    \includegraphics[width=0.45\linewidth]{content/diagrams/out/usecase/mietverträgeVerwalten/MietverträgeVerwalten.png}
    \caption{Usecase Mietverträge Verwalten}
  \end{center}
  \label{MietvertraegeVerwalten}
\end{figure}

\begin{figure}[H]
  \begin{center}
    \includegraphics[width=0.4\linewidth]{content/diagrams/out/usecase/mietzinsKontrollieren/MietzinsKontrollieren.png}
    \caption{Usecase Mietzins Kontrollieren}
  \end{center}
  \label{MietzinsKontrollieren}
\end{figure}

\begin{figure}[H]
  \begin{center}
    \includegraphics[width=0.4\linewidth]{content/diagrams/out/usecase/mahnungGenerieren/MahnungErstellen.png}
    \caption{Usecase Mahnung Erstellen}
  \end{center}
  \label{mahnung}
\end{figure}

\begin{figure}[H]
  \begin{center}
    \includegraphics[width=0.8\linewidth]{content/diagrams/out/usecase/rechnungErstellen/Rechnung erstellen.png}
    \caption{Usecase Rechnung erstellen}
  \end{center}
  \label{RechnungErstellen}
\end{figure}

\begin{figure}[H]
  \begin{center}
    \includegraphics[width=0.4\linewidth]{content/diagrams/out/usecase/kreditorErfassen/Kreditor erfassen.png}
    \caption{Usecase Kreditor Erfassen}
  \end{center}
  \label{kreditorErfassen}
\end{figure}

\begin{figure}[H]
  \begin{center}
    \includegraphics[width=0.4\linewidth]{content/diagrams/out/usecase/mieterErfassen/Mieter erfassen.png}
    \caption{Usecase Mieter erfassen}
  \end{center}
  \label{mieterErfassen}
\end{figure}

\subsection{Sequenzdiagramm}
\begin{figure}[H]
  \begin{center}
    \includegraphics[width=1\linewidth]{content/diagrams/out/sequenzdiagram/sequenzdiagram.png}
    \caption{Sequenzdiagramm}
    \label{sequenzdiagram}
  \end{center}
\end{figure}

\newpage
\subsection{Modellierung der Klassen}
\subsubsection{Klassendiagramm}
\begin{figure}[htbt]
  \begin{center}
    \includegraphics[width=0.9\linewidth]{content/diagrams/out/classdiagram/classdiagram.png}
    \caption{Klassendiagramm}
    \label{classdiagramm}
  \end{center}
\end{figure}

\subsubsection{Beschreibung der Fachklassen}
\begin{table}[H]
  \centering
  \settowidth\tymin{\textbf{Liegenschaft}}
  \setlength\extrarowheight{2pt}
    \begin{tabulary}{1.0\textwidth}{|L|m{15cm}|}
      \hline
      \rowcolor[HTML]{4473C5}\textbf{Klasse}& \textbf{Beschreibung}\\
    \hline
    \textbf{Persona} & Bildet alle Beteiligten Personen ab. Es werden die Mieter:in, Kreditor und die Hauswartungsperson über diese Klasse erfasst. Z.B. für den Mieter:in werden nicht alle Eigenschaften verwendet, wie auch für den Kreditor oder die Hauswartungsperson nicht alle Eigenschaften verwendet werden.\\
    \hline
    \textbf{Liegenschaft} & Bildet die Liegenschaft mit ihren Eigenschaften ab und beinhaltet die Hauswartungsperson als Persona Objekt. Jede Liegenschaft kann N-Objekte enthalten.\\
    \hline
    \textbf{Objekt} & Bildet das Objekt, welches in der Liegenschaft enthalten ist, mit seinen Eigenschaften ab. Jedes Objekt kann nur Eine Beziehungen zu einer Liegenschaft haben. Pro Objekt können aber N-Mietverträge vorhanden sein, wobei immer nur einer gültig sein darf. \\
    \hline
    \textbf{Mietvertrag} & Bildet den Mietvertrag mit seinen Eigenschaften ab. Jeder Mietvertrag kann einem Objekt zugeordnet sein. Der Mietvertrag ist entweder gültig oder ungültig.\\
    \hline
    \textbf{Rechnung} & Bildet die Rechnung mit ihren Eigenschaften ab. Es kann eine Rechnung auf ein Objekt oder auf eine Liegenschaft erstellt werden und einem Mieter oder einem Kreditor über ''Persona'' zugeordnet werden.\\
    \hline
    \textbf{User} & Bildet die User für das Login in der Applikation ab.\\
    \hline
    \textbf{Konto} & Bildet das Konto zum Verbuchen der Rechnung ab.\\
    \hline 
\end{tabulary}
\caption{Beschreibung der Fachklassen}
\end{table}

\subsection{Zustandsdiagramm}
\begin{figure}[H]
  \begin{center}
    \includegraphics[height=0.85\textheight]{content/diagrams/out/zustand/mietvertrag/mietvertrag.png}
    \caption{Zustand Mietvertrag}
    \label{zustMietvertrag}
  \end{center}
\end{figure}

\subsection{Modellierung der Datenbank}

\subsubsection{ERD}
\begin{figure}[H]
  \begin{center}
    \includegraphics[height=0.85\textheight]{content/diagrams/out/erd/erd.png}
    \caption{ERD}
    \label{ERD}
  \end{center}
\end{figure}

\subsubsection{Beschreibung der Fachentitäten}
\begin{table}[H]
  \centering
  \settowidth\tymin{\textbf{Liegenschaft}}
  \setlength\extrarowheight{2pt}
    \begin{tabulary}{1.0\textwidth}{|L|m{15cm}|}
      \hline
      \rowcolor[HTML]{4473C5}\textbf{Entität}& \textbf{Beschreibung}\\
    \hline
    \textbf{Liegenschaft} & Bildet die Liegenschaft ab. Hat eine 1:N Beziehungen zur Objekt- und zur Rechnungsentität.\\
    \hline
    \textbf{Objekt} & Bildet das Objekt ab. Das Objekt hate eine N:1 Beziehung zur Liegenschaft, beinhaltet also dessen Foreign-Key. Zusätzlich hat das Objekt noch eine 1:N-Beziehung zur Rechnung.\\
    \hline
    \textbf{Mietvertrag} & Bildet den Mietvertrag ab. Diese Entität beinhaltet den Foreign-Key von der Entität Persona zum Referenzieren des Mieters und den Foreign-Key zum Objekt, auf welches der Mietvertrag läuft. \\
    \hline
    \textbf{Persona} & Bildet alle Beteiligten Personen ab. Es werden die Mieter:in, Kreditor und die Hauswartungsperson über diese Entität erfasst. Somit ist je ein Foreign-Key in der Rechnung und im Mietvertrag enthalten. \\
    \hline
    \textbf{Rechnung} & Bildet die Rechnung ab. Da die Rechnung zur Liegenschaft und zum Objekt gestellt werden muss, ist je ein Foreign-Key von diesen Entitäten enthalten. Zum Verbuchen auf ein Konto wir der Foreign-Key der Entität Konto verwendet. Damit die Rechnung einem Mieter oder Kreditor gestellt werden kann, wird der Foreign-Key Persona verwendet.\\
    \hline
    \textbf{Konto} & Bildet das Konto zum Verbuchen der Rechnung ab und hat somit eine 1:N-Beziehung zur Rechnung\\
    \hline
    \textbf{User} & Bildet die User für das Login in der Applikation ab. Das Passwort wird als HASH gespeichert.\\
    \hline 
\end{tabulary}
\caption{Beschreibung der Fachentitäten}
\end{table}

\subsection{Systemarchitektur}
Die Programmiersprache für die Applikation ist C\# mit dem aktuellen .net Core 6 Framework. Das GUI wird mit WPF aufgebaut und die gesamte Struktur im MVVM-Pattern gehalten. Um das GUI ansprechend zu gestalten, wird das NuGet-Paket ''Material-Design'' verwendet.\\
Für das persistieren der Daten wird das Entity-Framework verwendet, mit welchem bereits in einem früheren Projekt Erfahrungen gesammelt werden konnten.
\subsection{Testkonzept} \label{testkonzept}
\subsubsection{Vorgehen}
\subsubsection{Testobjekte}
\subsubsection{Testfälle}
\subsection{Einführungskonzept}
\subsection{GUI-Design}