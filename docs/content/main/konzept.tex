\section{Konzept}

\subsection{Kontextdiagramm}
\begin{figure}[H]
  \begin{center}
    \includegraphics[width=0.5\linewidth]{content/diagrams/out/contextdiagram/context.png}
    \caption{Kontextdiagramm}
  \end{center}
  \label{contextdiag}
\end{figure}

\subsection{Geschäftsprozessanalyse}


\subsection{Detailanforderungen an das neue System}

\subsection{Use-Case's}
\begin{figure}[H]
  \begin{center}
    \includegraphics[width=0.6\linewidth]{content/diagrams/out/usecase/login/Login.png}
    \caption{Usecase Login}
  \end{center}
  \label{login}
\end{figure}

\begin{figure}[H]
  \begin{center}
    \includegraphics[width=0.5\linewidth]{content/diagrams/out/usecase/liegenschaftErfassen/LiegenschaftErfassen.png}
    \caption{Usecase Liegenschaft erfassen}
  \end{center}
  \label{Liegenschaft}
\end{figure}

\begin{figure}[H]
  \begin{center}
    \includegraphics[width=0.8\linewidth]{content/diagrams/out/usecase/objektErfassen/ObjektErfassen.png}
    \caption{Usecase Objekt Erfassen}
  \end{center}
  \label{objekt}
\end{figure}

\begin{figure}[H]
  \begin{center}
    \includegraphics[width=0.45\linewidth]{content/diagrams/out/usecase/mietverträgeVerwalten/MietverträgeVerwalten.png}
    \caption{Usecase Mietverträge Verwalten}
  \end{center}
  \label{MietvertraegeVerwalten}
\end{figure}

\begin{figure}[H]
  \begin{center}
    \includegraphics[width=0.4\linewidth]{content/diagrams/out/usecase/mietzinsKontrollieren/MietzinsKontrollieren.png}
    \caption{Usecase Mietzins Kontrollieren}
  \end{center}
  \label{MietzinsKontrollieren}
\end{figure}

\begin{figure}[H]
  \begin{center}
    \includegraphics[width=0.4\linewidth]{content/diagrams/out/usecase/mahnungGenerieren/MahnungErstellen.png}
    \caption{Usecase Mahnung Erstellen}
  \end{center}
  \label{mahnung}
\end{figure}

\begin{figure}[H]
  \begin{center}
    \includegraphics[width=0.8\linewidth]{content/diagrams/out/usecase/rechnungErstellen/Rechnung erstellen.png}
    \caption{Usecase Rechnung erstellen}
  \end{center}
  \label{RechnungErstellen}
\end{figure}

\begin{figure}[H]
  \begin{center}
    \includegraphics[width=0.4\linewidth]{content/diagrams/out/usecase/kreditorErfassen/Kreditor erfassen.png}
    \caption{Usecase Kreditor Erfassen}
  \end{center}
  \label{kreditorErfassen}
\end{figure}

\begin{figure}[H]
  \begin{center}
    \includegraphics[width=0.4\linewidth]{content/diagrams/out/usecase/mieterErfassen/Mieter erfassen.png}
    \caption{Usecase Mieter erfassen}
  \end{center}
  \label{mieterErfassen}
\end{figure}

\subsection{Zustandsdiagramme}
\subsection{Modellierunge der Datenbank}
\subsubsection{ERD}
\subsubsection{Beschreibung der Fachentitäten, Beziehungen und der referenziellen Integritätsbedingungen}
\subsection{Systemarchitektur}
Die Programmiersprache für die Applikation ist C\# mit der aktuelle .net Core 6. Das GUI wird mit WPF aufgebaut und die gesamte Struktur im MVVM-Pattern gehalten. Um das GUI ansprechend zu gestalten, wird das nuGet-Paket ''Material-Design'' verwendet.\\
Für das persistieren der Daten wird das Entity-Framework verwendet, mit welchem bereits in einem früheren Projekt Erfahrung gesammelt werden konnte.
\subsection{Testkonzept} \label{testkonzept}
\subsubsection{Vorgehen}
\subsubsection{Testobjekte}
\subsubsection{Testfälle}
\subsection{Einführungskonzept}
\subsection{GUI-Design}
