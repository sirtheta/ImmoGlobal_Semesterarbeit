\begin{table}[H]
  \newcolumntype{a}{>{\columncolor[HTML]{4473C5}}L}
  \centering
  \settowidth\tymin{\textbf{Kurzbeschreibung}}
  \setlength\extrarowheight{2pt}
  \begin{tabulary}{1.0\textwidth}{|a|m{12cm}|}
    \hline
    \textbf{Name}& Verwaltung von Objekten\\
    \hline 
    \textbf{Kurzbeschreibung} & Eine bestehendes Objekt muss editiert werden oder eine neues Objekt soll hinzugefügt werden \\
    \hline
    \textbf{Akteure} & Mitarbeiter in der Liegenschaftsverwaltung\\
    \hline
    \textbf{Auslöser} & Für die erfasste Liegenschaft wurden noch keine Objekte hinzugefügt\newline 
    Ein bestehendes Objekt muss ergänzt/korrigiert werden\\
    \hline
    \textbf{Ergebnis} & Das veränderte oder neu erstellte Objekt kann zu einer Liegenschaft hinzugefügt werden\\
    \hline
    \textbf{Eingehende Daten} & Informationen zum erfassenden Objekt\\
    \hline
    \textbf{Vorbedingungen} & Keine\\
    \hline
    \textbf{Nachbedingungen} & Keine\\
    \hline
    \textbf{Ablauf} & Der Benutzer trägt alle Muss-Daten für das neue Objekt ein und speichert es ab. \\
    \hline
  \end{tabulary}
  \caption{GA-Verwaltung von Objekten}
\end{table}

\begin{table}[H]
  \newcolumntype{a}{>{\columncolor[HTML]{4473C5}}L}
  \centering
  \settowidth\tymin{\textbf{Kurzbeschreibung}}
  \setlength\extrarowheight{2pt}
  \begin{tabulary}{1.0\textwidth}{|a|m{12cm}|}
    \hline
    \textbf{Name}& Verwaltung von Liegenschaften\\
    \hline 
    \textbf{Kurzbeschreibung} & Eine bestehende Liegenschaft muss editiert werden oder eine neue Liegenschaft soll hinzugefügt werden\\
    \hline
    \textbf{Akteure} & Mitarbeiter in der Liegenschaftsverwaltung\\
    \hline
    \textbf{Auslöser} & Die Verwaltung wird mit dem verwalten einer neuen Liegenschaft beauftragt \newline
    Bei einer Liegenschaft müssen Änderungen vorgenommen werden\\
    \hline
    \textbf{Ergebnis} & Die neuen Daten zu der Liegenschaft werden abgespeichert und in der Übersicht dargestellt\\
    \hline
    \textbf{Eingehende Daten} & Liegenschaft mit allen dazu gehörigen Informationen\\
    \hline
    \textbf{Vorbedingungen} & Die Daten zur Liegenschaft müssen bekannt sein\\
    \hline
    \textbf{Nachbedingungen} & Keine\\
    \hline
    \textbf{Ablauf} & Der Eingeloggte Mitarbeiter trägt alle Muss-Daten für die neue Liegenschaft ein und speichert diese. Der Mitarbeiter kann den Objekttyp für die Liegenschaft auswählen und die Informationen pro Objekt eintragen. Wenn der Objekttyp noch nicht erfasst wurde, muss der Objekttyp noch erstellt werden\newline
    Beim editieren einer Liegenschaft werden die Vorhandenen Bearbeitet. Falls ein neuer Objekttyp hinzugefügt werden muss der noch nicht in der Datenbank vorhanden ist, muss dieser wie bei einer neuen Liegenschaft, hinzugefügt werden\\
    \hline
  \end{tabulary}
  \caption{GA-Verwaltung von Liegenschaften}
\end{table}

\begin{table}[H]
  \newcolumntype{a}{>{\columncolor[HTML]{4473C5}}L}
  \centering
  \settowidth\tymin{\textbf{Kurzbeschreibung}}
  \setlength\extrarowheight{2pt}
  \begin{tabulary}{1.0\textwidth}{|a|m{12cm}|}
    \hline
    \textbf{Name}& Übernahme- und Übergabeprotokoll der Mietobjekte\\
    \hline 
    \textbf{Kurzbeschreibung} &\\
    \hline
    \textbf{Akteure} & Mitarbeiter in der Liegenschaftsverwaltung, Hauswartungsperson, Mieter\\
    \hline
    \textbf{Auslöser} & Beginn- oder Ende eines Mietverhältnisses\\
    \hline
    \textbf{Ergebnis} & Ein Übernahme- und Übergabeprotokoll für das Mietobjekt\\
    \hline
    \textbf{Eingehende Daten} &  
    \begin{itemize}
      \item Kündigung
      \item Mietvertrag
      \item Bei Übergabe zurück an die Verwaltung das Übernahmeprotokoll
    \end{itemize}\\
    \hline
    \textbf{Vorbedingungen} & Keine\\
    \hline
    \textbf{Nachbedingungen} & Keine\\
    \hline
    \textbf{Ablauf} & Für die Übernahme und die Übergabe wird von dem zuständigen Mitarbeiter das Protokoll ausgefüllt und von den Mietparteien auf Vollständigkeit geprüft und unterschrieben. Der Prozess findet komplett Digital statt. Unterschrieben wird das Protokoll auf einem unterschriftsfähigen Tablet. Anschliessend wird daraus ein PDF (PDF/A) generiert und dem Mieter wahlweise per E-Mail oder ausgedruckt per Post zugesandt\\
    \hline
  \end{tabulary}
  \caption{GA-Übernahme- und Übergabeprotokoll der Mietobjekte}
\end{table}

\begin{table}[H]
  \newcolumntype{a}{>{\columncolor[HTML]{4473C5}}L}
  \centering
  \settowidth\tymin{\textbf{Kurzbeschreibung}}
  \setlength\extrarowheight{2pt}
  \begin{tabulary}{1.0\textwidth}{|a|m{12cm}|}
    \hline
    \textbf{Name}& Verwaltung der Mieter:innen\\
    \hline 
    \textbf{Kurzbeschreibung} & Neue Mieter:innen werden in der Applikation hinzugefügt oder bestehende Mieter:innen bearbeitet\\
    \hline
    \textbf{Akteure} & Mitarbeiter in der Liegenschaftsverwaltung, Mieter\\
    \hline
    \textbf{Auslöser} & 
    \begin{itemize}
      \item Vermietung eines Objektes an einen neuen Mieter:in
      \item Namensänderungen
      \item Zivilstandänderungen
      \item Geburten
    \end{itemize}\\
    \hline
    \textbf{Ergebnis} & \begin{itemize}
      \item Ein neuer Mieter:in wurde in der Applikation erfasst
      \item Änderung an bestehendem Mieter:in wurden durchgeführt
    \end{itemize}\\
    \hline
    \textbf{Eingehende Daten} & Informationen über den Mieter:in\\
    \hline
    \textbf{Vorbedingungen} & Die Informationen über den Mieter:in müssen bekannt sein\\
    \hline
    \textbf{Nachbedingungen} & Keine\\
    \hline
    \textbf{Ablauf} & Der Benutzer der Applikation erstellt einen neuen Mieter:in. Er gibt alle Informationen ein und Speichert die Daten ab. Bei einer Änderung eines bestehenden Mieter:in, wird dieser in der Applikation gesucht und ausgewählt, editiert und abgespeichert\\
    \hline
  \end{tabulary}
  \caption{GA-Verwaltung der Mieter:innen}
\end{table}

\begin{table}[H]
  \newcolumntype{a}{>{\columncolor[HTML]{4473C5}}L}
  \centering
  \settowidth\tymin{\textbf{Kurzbeschreibung}}
  \setlength\extrarowheight{2pt}
  \begin{tabulary}{1.0\textwidth}{|a|m{12cm}|}
    \hline
    \textbf{Name}& Erstellen / Verwalten der Mietverträge\\
    \hline 
    \textbf{Kurzbeschreibung} & \\
    \hline
    \textbf{Akteure} & Mitarbeiter in der Liegenschaftsverwaltung, Mieter:in\\
    \hline
    \textbf{Auslöser} & Bewerbung eines neuen Mieters wurde akzeptiert\\
    \hline
    \textbf{Ergebnis} & Gültiger (unterschriebener) Mietvertrag\\
    \hline
    \textbf{Eingehende Daten} & 
    \begin{itemize}
      \item Bewerbung mit allen Daten zum zukünftigen Mieter
      \item Betreibungsregisterauszug
    \end{itemize}\\
    \hline
    \textbf{Vorbedingungen} & 
    \begin{itemize}
      \item Bewerbung des Mieters wurde akzeptiert
      \item Bonität des Mieters wurde überprüft
    \end{itemize}\\
    \hline
    \textbf{Nachbedingungen} & Keine\\
    \hline
    \textbf{Ablauf} & Der Benutzer der Applikation erstellt im betreffenden Mietobjekt einen neuen Mietvertrag. Den Mieter wählt er aus der Liste der vorhandenen Mieter aus. Falls der neue Mieter noch nicht erfasst wurde, kann er diesen erstellen. Anschliessend wird der neue Mietvertrag ausgedruckt und kann von beiden Parteien unterschrieben werden. Nachdem der Mietvertrag unterschrieben ist, wird dieser als PDF in der Applikation abgelegt und als Gültig markiert. Bei einer Kündigung des Mietverhältnisses wird die Kündigung des Mietvertrags in die Applikation hochgeladen und der der Status auf gekündigt geändert. Sobald dann das ende des Mietverhältnisses erreicht ist, wird der Status auf beendet gesetzt.\\
    \hline
  \end{tabulary}
  \caption{GA-Erstellen / Verwalten der Mietverträge}
\end{table}

\begin{table}[H]
  \newcolumntype{a}{>{\columncolor[HTML]{4473C5}}L}
  \centering
  \settowidth\tymin{\textbf{Kurzbeschreibung}}
  \setlength\extrarowheight{2pt}
  \begin{tabulary}{1.0\textwidth}{|a|m{12cm}|}
    \hline
    \textbf{Name}& Erfassen von Ein- und Ausgaben\\
    \hline 
    \textbf{Kurzbeschreibung} & Alle Einnahmen und alle Ausgaben werden erfasst\\
    \hline
    \textbf{Akteure} & Mitarbeiter in der Liegenschaftsverwaltung\\
    \hline
    \textbf{Auslöser} & Zahlungs-Eingang / Ausgang\\
    \hline
    \textbf{Ergebnis} & Neuer Zahlungs-Ein/Ausgang ist im System erfasst\\
    \hline
    \textbf{Eingehende Daten} &
    \begin{itemize}
      \item Rechnungen
      \item Abrechnungen
    \end{itemize} \\
    \hline
    \textbf{Vorbedingungen} & Keine\\
    \hline
    \textbf{Nachbedingungen} & Keine\\
    \hline
    \textbf{Ablauf} & Der Benutzer trägt alle Zahlungseingänge im System ein. Wenn eine Zahlung anhand einer Rechnung getätigt werden muss, bezahlt er diese und trägt den Rechnungsbetrag ebenfalls im System ein\\
    \hline
  \end{tabulary}
  \caption{GA-Kreditor erfassen}
\end{table}

\begin{table}[H]
  \newcolumntype{a}{>{\columncolor[HTML]{4473C5}}L}
  \centering
  \settowidth\tymin{\textbf{Kurzbeschreibung}}
  \setlength\extrarowheight{2pt}
  \begin{tabulary}{1.0\textwidth}{|a|m{12cm}|}
    \hline
    \textbf{Name}&Mietzins- und Nebenkostenkontrolle\\
    \hline 
    \textbf{Kurzbeschreibung} & Der Eingang der Mietzinse und der Nebenkostenzahlungen wird kontrolliert und bei fehlenden Eingängen wird eine Mahnung erstellt\\
    \hline
    \textbf{Akteure} & Mitarbeiter in der Liegenschaftsverwaltung\\
    \hline
    \textbf{Auslöser} & Mietzins- und Nebenkostenzahlungsüberprüfung\\
    \hline
    \textbf{Ergebnis} & Bei negativer Prüfung eine Mahnung\newline 
    Bei positiver Prüfung nichts\\
    \hline
    \textbf{Eingehende Daten} & Mietvertrag\\
    \hline
    \textbf{Vorbedingungen} & Es müssen gültige Mietverträge im System erfasst sein \\
    \hline
    \textbf{Nachbedingungen} & Keine \\
    \hline
    \textbf{Ablauf} & Der Benutzer überprüft ob für ein entsprechendes Objekte/Liegenschaft die Miete und die Nebenkosten bezahlt wurde. Wenn die Miete oder die Nebenkosten noch nicht bezahlt wurde, kann er direkt eine vorgefertigte Mahnung generieren und ausdrucken.\\
    \hline
  \end{tabulary}
  \caption{GA-Mietzins- und Nebenkostenkontrolle}
\end{table}


\begin{table}[H]
  \newcolumntype{a}{>{\columncolor[HTML]{4473C5}}L}
  \centering
  \settowidth\tymin{\textbf{Kurzbeschreibung}}
  \setlength\extrarowheight{2pt}
  \begin{tabulary}{1.0\textwidth}{|a|m{12cm}|}
    \hline
    \textbf{Name}&Rechnung erstellen\\
    \hline 
    \textbf{Kurzbeschreibung} & Eine neue Rechnung erstellen\\
    \hline
    \textbf{Akteure} & Mitarbeiter in der Liegenschaftsverwaltung\\
    \hline
    \textbf{Auslöser} & Es wurde eine Dienstleistung, die die Liegenschaftsverwaltung getätigt hat, bezogen \newline 
    Für eine Liegenschaft oder ein Objekt wurde die Nebenkostenabrechnung erstellt\\
    \hline
    \textbf{Ergebnis} & Eine Rechnung\\
    \hline
    \textbf{Eingehende Daten} & Informationen warum die Rechnung erstellt werden muss \\
    \hline
    \textbf{Vorbedingungen} & Keine\\
    \hline
    \textbf{Nachbedingungen} & Keine\\
    \hline
    \textbf{Ablauf} & Der Benutzer erstellt eine neue Rechnung und gibt alle nötigen Informationen für die Rechnung ein.\\
    \hline
  \end{tabulary}
  \caption{GA-Rechnung erstellen}
\end{table}