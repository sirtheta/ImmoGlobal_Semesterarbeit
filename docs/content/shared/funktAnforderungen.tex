\begin{table}[ht]
  \centering
  \settowidth\tymin{\textbf{Prio}}
  \setlength\extrarowheight{2pt}
  \begin{tabulary}{1.0\textwidth}{|m{16cm}|L|}
    \hline
    \textbf{Übersicht der Mietobjekte}&\textbf{Prio}\\
    \hline
    Dem Benutzer muss beim öffnen der Applikation eine Übersicht der Mitobjekte zur verfügung stehen. Es wird pro Objekt angezeigt, ob das Objekt vermietet ist oder nicht. & 1\\ 
    \hline
  \end{tabulary}
  \caption{AF-1.1}
  \label{af1.1}
\end{table}

\begin{table}[h]
  \centering
  \settowidth\tymin{\textbf{Prio}}
  \setlength\extrarowheight{2pt}
  \begin{tabulary}{1.0\textwidth}{|m{16cm}|L|}
    \hline
    \textbf{Mietverträge}&\textbf{Prio}\\
    \hline
    Die Applikation muss die Mietverträge mit folgenden Angaben verwalten können:
    \begin{itemize}
      \item Mieter:in
      \item Liegenschaft / Objekt
      \item Mietbeginn und gegebenenfalls Mietende
      \item Monatlicher Mietzins, monatliche Nebenkosten
      \item Art der Nebenkosten (Pauschal bzw. Akonto)
      \item Auflistung der Kostenarten, welche über die Nebenkosten abgerechnet werden (Heiz-, Warmwasseraufbereitungs-, Wasser-, Hauswart-, Treppen-reinigungs-, Gartenarbeits-, Strom-, Lift-, Kabelfernsehkosten sowie Abwasser- Kehrichtgebühren)
      \item Mietdepot (Ja/Nein) und Depotbetrag in CHF
    \end{itemize}  & 1\\ 
    \hline
  \end{tabulary}
  \caption{AF-1.2}
  \label{af12}
\end{table}

\begin{table}[h]
  \centering
  \settowidth\tymin{\textbf{Prio}}
  \setlength\extrarowheight{2pt}
  \begin{tabulary}{1.0\textwidth}{|m{16cm}|L|}
    \hline
    \textbf{Übergabe Mietobjekte}&\textbf{Prio}\\
    \hline
      Protokollieren der Übergabe an die neuen Mieter & 1\\
    \hline
  \end{tabulary}
  \caption{AF-1.3}
  \label{af13}
\end{table}

\begin{table}[H]
  \centering
  \settowidth\tymin{\textbf{Prio}}
  \setlength\extrarowheight{2pt}
  \begin{tabulary}{1.0\textwidth}{|m{16cm}|L|}
    \hline
    \textbf{Mietzinsinkasso}&\textbf{Prio}\\
    \hline
      Diese Funktion stellt sicher dass der Benutzer kontrollieren kann ob die Miete bezahlt wurde. Falls die Miete ausstehend ist, kann hier eine Mahnung ausgelöst werden. & 1\\
    \hline
  \end{tabulary}
  \caption{AF-1.4}
  \label{af14}
\end{table}

\begin{table}[H]
  \centering
  \settowidth\tymin{\textbf{Prio}}
  \setlength\extrarowheight{2pt}
  \begin{tabulary}{1.0\textwidth}{|m{16cm}|L|}
    \hline
    \textbf{Nebenkostenabrechnung}&\textbf{Prio}\\
    \hline
      Es muss eine Nebenkostenabrechnung erstellt werden können. & 1\\
    \hline
  \end{tabulary}
  \caption{AF-1.5}
  \label{af15}
\end{table}

\begin{table}[H]
  \centering
  \settowidth\tymin{\textbf{Prio}}
  \setlength\extrarowheight{2pt}
  \begin{tabulary}{1.0\textwidth}{|m{16cm}|L|}
    \hline
    \textbf{Einfache Buchhaltung}&\textbf{Prio}\\
    \hline
      Es muss eine einfache Buchhaltung mit Ein- und Ausgaben geführt werden können. & 1\\
    \hline
  \end{tabulary}
  \caption{AF-1.6}
  \label{af16}
\end{table}

\begin{table}[H]
  \centering
  \settowidth\tymin{\textbf{Prio}}
  \setlength\extrarowheight{2pt}
  \begin{tabulary}{1.0\textwidth}{|m{16cm}|L|}
    \hline
    \textbf{Instandhaltung der Gebäude}&\textbf{Prio}\\
    \hline
      Extern in Auftrag gegebene Instandhaltungsaufträge müssen verwaltet werden können. & 1\\
    \hline
  \end{tabulary}
  \caption{AF-1.7}
  \label{af17}
\end{table}

\begin{table}[h]
  \centering
  \settowidth\tymin{\textbf{Prio}}
  \setlength\extrarowheight{2pt}
  \begin{tabulary}{1.0\textwidth}{|m{16cm}|L|}
    \hline
    \textbf{Rücknahme der Mietobjekte}&\textbf{Prio}\\
    \hline
      Bei Rücknahme der Mietobjekte muss ein Übernahmeprotokoll erstellt werden können & 1\\
    \hline
  \end{tabulary}
  \caption{AF-1.8}
  \label{af18}
\end{table}
