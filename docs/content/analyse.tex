\part{Pflichtenheft}
\section{Analyse}
\subsection{Projektabgrenzung}
Während der Semesterarbeit soll ein Prototyp einer Applikation wie in den Anforderungen in Kapitel \ref{musskrit} beschrieben, entstehen. Der Prototyp wird so aufgebaut, dass die Daten nur Lokal in einer Datenbank verwaltet werden. Auf den Anspruch, dass die Liegenschaftsverwaltung ImmoGlobal mehrere Standorte hat die auf die Applikation zugreifen müssen, wird in der Prototypphase nicht eingegangen.

\subsection{Ziele}
\begin{itemize}
  \item Die Applikation deckt alle Muss-Ziele am Abgabetermin 10.5.2022, wie in den Anforderungen in Kapitel \ref{musskrit} beschrieben, ab.
  \item Die Applikation kann die Daten persistent abspeichern.
\end{itemize}

\subsection{Anforderungsanalyse}
\subsubsection{Funktionale Anforderungen}
\subsubsection{Nichtfunktionale Anforderungen}

\subsection{Stakeholder Analyse}
Die verschiedenen Stakeholder an diesem Projekt haben verschiedene Interessen, die es während dem gesamten Projekt zu berücksichtigen gilt. Als Stakeholder sind all jene gemeint, welche direkt oder indirekt mit dem System in Kontakt stehen. Besonderes Augenmerk sollte auf die Stakeholder welche direkt mit dem System in Verbindung kommen werden, gelegt werden.
Um die verschiedenen Stakeholder aufzuzeigen, werden diese in 2 verschiedene Gruppen eingeteilt.

\subsubsection{Stakeholder mit direkten Intetressen}
Diese Gruppe beinhaltet die Stakeholder, welche direkt mit der Applikation arbeiten werden.

\begin{itemize}
  \item Auftraggeber
  \item Mitarbeiter:innen
  \item Hauswartungspersonen
  \item Investoren/Eigentümer der Mietobjekte
\end{itemize}

\subsubsection{Stakeholder mit indirekten Intetressen}

\begin{itemize}
  \item Gesetzgeber und Behörden
  \item Mietparteien
  \item Lieferanten
\end{itemize}

\newpage
\subsection{Stakeholder Risikoanalyse}
Damit das Risiko besser eingeschätzt werden kann, werden die Kriterien «Macht und Einfluss» und «Interesse» bewertet. Durch Multiplikation dieser Faktoren wird das Risiko berechnet.


\begin{table}[h]
  \centering
  \settowidth\tymin{\textbf{Stakeholder}}
  \setlength\extrarowheight{2pt}
  \begin{tabulary}{1.0\textwidth}{|L|L|L|L|L|L|}
    \hline
    \textbf{Stakeholder} & 
    \textbf{Beschreibung} & 
    \textbf{Erwartung}& 
    \textbf{Macht und Einfluss} & 
    \textbf{Interesse} & 
    \textbf{Risiko}\\
    \hline
    Auftraggeber & 
    Geschäftsführung von ImmoGlobal& 
    Einfache, dem Bedürfnis von ImmoGlobal entsprechendem Software welche auf das Wachstum der Firma angepasst ist. &
    7 &
    5 & 
    35\\ 
    \hline
    Mitarbeiter:innen & 
    Mitarbeiter:innen von ImmoGlobal an den 3 Standorten & 
    Einfache, leicht zu bedienende Software&
    8 &
    8 &
    64\\
    \hline
    Hauswartungs-personen & 
    Hauswarte im Dienst von ImmoGlobal &
    Einfaches auffinden der Objekte durch eindeutige Kennzeichnung &
    5 &
    6 &
    30 \\
    \hline
    Investoren/ Eigentümer der Mietobjekte & 
    Personen/ Juristische Personen welchen die Immobilie gehört &
    Keine Besonderen Erwartungen, Kosten dürfen nicht erhöht werden &
    1 &
    1 &
    1 \\
    \hline
    Gesetzgeber und Behörden &
    Mietgesetz, Steuerbehörde &
    Gesetze werden eingehalten &
    5 &
    1 &
    5 \\
    \hline
    Mietparteien & 
    Mieter der Objekte von ImmoGlobal &
    Korrekte Übergabeprotokolle &
    1 &
    2 &
    2 \\
    \hline
    Lieferanten & 
    Firmen, welche im Aufträge von ImmoGlobal ausführen &
    Schnelle Abwicklung der Zahlungen &
    1 &
    2 &
    2 \\
    \hline
  \end{tabulary}
  \caption{Stakeholder Risikoanalyse}
  \label{tblRisikonalyse}
\end{table}

Wie aus der Risikoanalyse hervorgeht, ist besonders auf die Nutzer der Software zu achten. Das Risiko ist bei den Nutzern vor allem die Akzeptanz der Software. Um dieses Risiko möglichst gering zu halten, müssen die zukünftigen Nutzer von Anfang an in den Entwicklungsprozess mittels WireFrames und ersten Prototypen Miteingebungen werden. 
