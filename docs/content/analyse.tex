\part{Pflichtenheft}
\section{Analyse}
\subsection{Projektabgrenzung}
Während der Semesterarbeit soll ein Prototyp einer Applikation wie in den Anforderungen in Kapitel \ref{musskrit} beschrieben, entstehen. Der Prototyp wird so aufgebaut, dass die Daten nur Lokal in einer Datenbank verwaltet werden. Auf den Anspruch, dass die Liegenschaftsverwaltung ImmoGlobal mehrere Standorte hat die auf die Applikation zugreifen müssen, wird in der Prototypphase nicht eingegangen.

\subsection{Ziele}
\begin{itemize}
    \item Die Applikation deckt alle Muss-Ziele am Abgabetermin 10.5.2022, wie in den Anforderungen in Kapitel \ref{musskrit} beschrieben, ab.
    \item Die Applikation kann die Daten persistent abspeichern.
\end{itemize}

\subsection{Stakeholder Analyse}
Die verschiedenen Stakeholder an diesem Projekt haben verschiedene Interessen, die es während dem gesamten Projekt zu berücksichtigen gilt.
Um die verschiedenen Stakeholder aufzuzeigen, werden diese in 2 verschiedene Gruppen eingeteilt.

\subsubsection{Stakeholder mit direkten Intetressen}
Diese Gruppe beinhaltet die Stakeholder, welche direkt mit der Applikation arbeiten werden.

\begin{itemize}
    \item Auftraggeber
    \item Mitarbeiter:innen
    \item Hauswartungspersonen
    \item Investoren/Eigentümer der Mietobjekte
\end{itemize}

\subsubsection{Stakeholder mit indirekten Intetressen}

\begin{itemize}
    \item Gesetzgeber und Behörden
    \item Mietparteien
    \item Lieferanten
\end{itemize}

\subsection{Risikoanalyse}

\subsection{Testfälle}


\subsection{Anforderungsanalyse}
\subsubsection{Funktionale Anforderungen}
\subsubsection{Nichtfunktionale Anforderungen}
