\section{Vorwort}
Dieses Dokument dient als Pflichtenheft für die Semesterarbeit im Fach System- und Softwareengineering. Die Aufgabe der Semesterarbeit ist es eine Applikation für eine Immobilienverwaltung zu erstellen. Das Pflichtenheft dient als Übersicht der Anforderungen und des geeigneten Vorgehensmodells sowie zur Definition der Werkzeuge und Technologien, mit welchen das Projekt realisiert werden kann.

\begin{table}[h]
  \centering
  \setlength\extrarowheight{2pt}
  \begin{tabulary}{1.0\textwidth}{|L|L|L|L|}
    \hline
    \textbf{Rev.}&\textbf{Datum}&\textbf{Autor}&\textbf{Änderungen}\\
    \hline
    1.0 & 25.02.2022 & \autor & Vorlage, Struktur erstellt\\
    \hline
    1.1 & 06.03.2022 & \autor & \\
    \hline
  \end{tabulary}
  \caption{Dokumentversion}
  \label{version}
\end{table}
\newpage

\section{Einleitung}
% Hier werden die Dienste beschrieben, die für den Benutzer bereitgestellt werden. Die nichtfunktionalen Anforderungen sollten eben falls in diesem Abschnitt beschrieben werden. Diese Beschreibung kann natürliche Sprache, Diagramme oder andere für die Kunden  verständliche Notationen benutzen. Produkt- und Entwicklungsstandards, die befolgt werden müssen, sollten ebenfalls festgelegt werden.

\subsection{Ausgangslage}

Die Firma ImmoGlobal verwaltet im Auftrag ihrer Kunden Wohnliegenschaften in der ganzen Schweiz. Sie beschäftigt insgesamt 30 Personen in der Administration, 20 davon am Hauptsitz in Bern und je 5 in Lausanne und Zürich. Hinzu kommen Hauswartungspersonen, die für eine oder mehrere Liegenschaften zuständig sind. Die Aktivitäten im Tessin werden von Zürich aus gesteuert.
Das Aufgabenspektrum von ImmoGlobal umfasst folgende Aktivitäten:
\begin{itemize}
  \item Vermietung der verschiedenen Mietobjekte (Wohnungen, Räume, Garagen und Parkplätze)
  \item Erstellung und Verwaltung der Mietverträge.
  \item Übergabe der Mietobjekte an die neuen Mieter (inkl. Übergabeprotokoll)
  \item Mietzinsinkasso (inkl. Kontrolle und Mahnung)
  \item Abrechnung der Nebenkosten
  \item Führen einer einfachen Buchhaltung der Ein- und Ausgaben
  \item Instandhaltung der Gebäude (Reparaturen und Renovationsarbeiten werden extern in Auftrag gegeben)
  \item Rücknahme der Mietobjekte nach Ablauf des Mietverhältnisses (inkl. Übernahmeprotokoll)
\end{itemize}

\subsection{Situationsanalyse (IST-Zustand)}
ImmoGlobal arbeitet heute mit einer veralteten, eigenentwickelten Software sowie mit den gängigen Microsoft Office-Produkten (Word, Excel und Outlook). Diese Lösung ist nicht mehr zeitgerecht und muss angesichts des Wachstums der Firma in naher Zukunft ersetzt werden. Der Geschäftsführer von ImmoGlobal hat verschiedene, auf dem Markt verfügbare, Standardsoftware für die Verwaltung von Liegenschaften an-geschaut, erachtet sie aber als zu kompliziert. Er bevorzugt eine einfache, auf die Bedürfnisse von ImmoGlobal zugeschnittene Lösung und beauftragt Sie deshalb eine solche Software zu entwickeln.\\
Die Software soll den Geschäftsführer:innen und seinen Mitarbeiter:innen einen besseren Überblick über die zu verwaltenden Liegenschaften und Objekte erlauben, die Effizienz steigern sowie die zuhanden der Kunden zu liefernden Abrechnungen automatisieren.\\
Basierend auf den Angaben im vorliegenden Dokument ist zuerst ein Pflichtenheft zu erstellen. Dabei sind die vorliegenden Informationen auf ihre Vollständigkeit hin zu überprüfen. Allfällige Lücken müssen im Hinblick auf das Verfassen des Pflichtenhefts geschlossen werden. Anschliessend ist auf der Basis des von Ihnen erstellten Pflichtenhefts ein Prototyp der Applikation zu entwickeln.

\subsection{Anforderungen}
Jede Liegenschaft wird durch eine eindeutige Nummer gekennzeichnet und beinhaltet verschiedene Objekte (Wohnungen, Räume, Garagen oder Parkplätze), welche wiederum eine eindeutige Kennzeichnung innerhalb der Liegenschaft aufweisen.\\
Für jedes Objekt können mehrere Mietverträge im System vorhanden sein, wobei zu einem bestimmten Zeitpunkt nur einer gültig sein darf. Für jeden Mietvertrag werden die geschuldeten und effektiv bezahlten Mietzinse und Nebenkostenanteile auf ent-sprechenden Konten gebucht, so dass eine Übersicht über die ausstehenden Beträge jederzeit erstellt werden kann.\\
Bei den Kreditorenrechnungen wird zwischen solchen, die ein bestimmtes Objekt betreffen, und solchen, welche die Liegenschaft als Ganzes betreffen, unterschieden. Dementsprechend muss jede Rechnung entweder einem Objekt oder einer Liegen-schaft zugeordnet und auf ein entsprechendes Konto gebucht werden.\\
\subsection{Rahmenbedingungen}
Die Entwicklungsumgebung für dieses Projekt entspricht derjenigen, welche während des Programmierunterrichts verwendet wurde. Begründete Abweichungen davon sind erlaubt, sofern der damit verbundene Eigenprogrammieraufwand vergleichbar bleibt. Die Gründe dafür müssen in der Projektdokumentation dargelegt werden.
Die Semesterarbeit wird als Informatikprojekt durchgeführt. Es wird erwartet, dass die gelernten Methoden und Instrumente des Projektmanagements sowie des System Engineerings in die Praxis umgesetzt werden. Dazu gehören insbesondere (aber nicht nur) die Wahl eines adäquaten Vorgehensmodells, eine fundierte und detaillierte Planung der verschiedenen Aktivitäten sowie eine vollständige und nachvollziehbare Dokumentation des Projekts. Analyse und Design werden mittels UML vorgenommen.
Das Projektmanagement umfasst nebst der bereits aufgeführten detaillierten Planung noch ein entsprechendes Controlling, eine Risikoanalyse mit Massnahmen, ein Qualitätsmanagement sowie ein Konfigurationsmanagement in einer sinnvollen Granularität.

\subsection{Ziele}
\begin{itemize}
  \item Die Applikation deckt alle Muss-Ziele am Abgabetermin 10.5.2022, wie in den Anforderungen beschrieben, ab.
  \item Die Applikation kann die Daten persistent abspeichern.
\end{itemize}

\subsection{Abgrenzung}
Während der Semesterarbeit soll ein Prototyp einer Applikation wie in den Anforderungen beschrieben, entstehen. Der Prototyp wird so aufgebaut, dass die Daten nur Lokal in einer Datenbank verwaltet werden. Auf den Anspruch, dass die Liegenschaftsverwaltung ImmoGlobal mehrere Standorte hat, die auf die Applikation zugreifen müssen, wird in der Prototypphase nicht eingegangen.\\
Die Verwaltung der Mitarbeiter gehört nicht zum Umfang dieser Arbeit.

\newpage

\section{Definition der Benutzeranforderungen}

\subsection{Funktionale Anforderungen}

\begin{table}[ht]
  \centering
  \settowidth\tymin{\textbf{Prio}}
  \setlength\extrarowheight{2pt}
  \begin{tabulary}{1.0\textwidth}{|m{16cm}|L|}
    \hline
    \textbf{Übersicht der Mietobjekte}&\textbf{Prio}\\
    \hline
    Dem Benutzer muss beim öffnen der Applikation eine Übersicht der Mitobjekte zur verfügung stehen. Es wird pro Objekt angezeigt, ob das Objekt vermietet ist oder nicht. & 1\\ 
    \hline
  \end{tabulary}
  \caption{AF-1.1}
  \label{af1.1}
\end{table}

\begin{table}[h]
  \centering
  \settowidth\tymin{\textbf{Prio}}
  \setlength\extrarowheight{2pt}
  \begin{tabulary}{1.0\textwidth}{|m{16cm}|L|}
    \hline
    \textbf{Mietverträge}&\textbf{Prio}\\
    \hline
    Die Applikation muss die Mietverträge mit folgenden Angaben verwalten können:
    \begin{itemize}
      \item Mieter:in
      \item Liegenschaft / Objekt
      \item Mietbeginn und gegebenenfalls Mietende
      \item Monatlicher Mietzins, monatliche Nebenkosten
      \item Art der Nebenkosten (Pauschal bzw. Akonto)
      \item Auflistung der Kostenarten, welche über die Nebenkosten abgerechnet werden (Heiz-, Warmwasseraufbereitungs-, Wasser-, Hauswart-, Treppen-reinigungs-, Gartenarbeits-, Strom-, Lift-, Kabelfernsehkosten sowie Abwasser- Kehrichtgebühren)
      \item Mietdepot (Ja/Nein) und Depotbetrag in CHF
    \end{itemize}  & 1\\ 
    \hline
  \end{tabulary}
  \caption{AF-1.2}
  \label{af12}
\end{table}

\begin{table}[h]
  \centering
  \settowidth\tymin{\textbf{Prio}}
  \setlength\extrarowheight{2pt}
  \begin{tabulary}{1.0\textwidth}{|m{16cm}|L|}
    \hline
    \textbf{Übergabe Mietobjekte}&\textbf{Prio}\\
    \hline
      Protokollieren der Übergabe an die neuen Mieter & 1\\
    \hline
  \end{tabulary}
  \caption{AF-1.3}
  \label{af13}
\end{table}

\begin{table}[H]
  \centering
  \settowidth\tymin{\textbf{Prio}}
  \setlength\extrarowheight{2pt}
  \begin{tabulary}{1.0\textwidth}{|m{16cm}|L|}
    \hline
    \textbf{Mietzinsinkasso}&\textbf{Prio}\\
    \hline
      Diese Funktion stellt sicher dass der Benutzer kontrollieren kann ob die Miete bezahlt wurde. Falls die Miete ausstehend ist, kann hier eine Mahnung ausgelöst werden. & 1\\
    \hline
  \end{tabulary}
  \caption{AF-1.4}
  \label{af14}
\end{table}

\begin{table}[H]
  \centering
  \settowidth\tymin{\textbf{Prio}}
  \setlength\extrarowheight{2pt}
  \begin{tabulary}{1.0\textwidth}{|m{16cm}|L|}
    \hline
    \textbf{Nebenkostenabrechnung}&\textbf{Prio}\\
    \hline
      Es muss eine Nebenkostenabrechnung erstellt werden können. & 1\\
    \hline
  \end{tabulary}
  \caption{AF-1.5}
  \label{af15}
\end{table}

\begin{table}[H]
  \centering
  \settowidth\tymin{\textbf{Prio}}
  \setlength\extrarowheight{2pt}
  \begin{tabulary}{1.0\textwidth}{|m{16cm}|L|}
    \hline
    \textbf{Einfache Buchhaltung}&\textbf{Prio}\\
    \hline
      Es muss eine einfache Buchhaltung mit Ein- und Ausgaben geführt werden können. & 1\\
    \hline
  \end{tabulary}
  \caption{AF-1.6}
  \label{af16}
\end{table}

\begin{table}[H]
  \centering
  \settowidth\tymin{\textbf{Prio}}
  \setlength\extrarowheight{2pt}
  \begin{tabulary}{1.0\textwidth}{|m{16cm}|L|}
    \hline
    \textbf{Instandhaltung der Gebäude}&\textbf{Prio}\\
    \hline
      Extern in Auftrag gegebene Instandhaltungsaufträge müssen verwaltet werden können. & 1\\
    \hline
  \end{tabulary}
  \caption{AF-1.7}
  \label{af17}
\end{table}

\begin{table}[h]
  \centering
  \settowidth\tymin{\textbf{Prio}}
  \setlength\extrarowheight{2pt}
  \begin{tabulary}{1.0\textwidth}{|m{16cm}|L|}
    \hline
    \textbf{Rücknahme der Mietobjekte}&\textbf{Prio}\\
    \hline
      Bei Rücknahme der Mietobjekte muss ein Übernahmeprotokoll erstellt werden können & 1\\
    \hline
  \end{tabulary}
  \caption{AF-1.8}
  \label{af18}
\end{table}


\subsection{Nichtfunktionale Anforderungen}

\input{content/shared/nichtFunktAnforderungen.tex}

\newpage
\section{Systemarchitektur}
% Dieses Kapitel gibt einen groben Überblick über die erwartete Systemarchitektur, der die Verteilung der Funktionen auf die Systemmodule zeigt. Wiederverwendete Komponenten sollten gekennzeichnet werden.
Die Programmiersprache für die Applikation ist C\# mit der aktuelle .net Core 6. Das GUI wird mit WPF aufgebaut und die gesamte Struktur im MVVM-Pattern gehalten. Um das GUI ansprechend zu gestalten, wird das nuGet-Paket ''Material-Design'' verwendet.\\
Für das persistieren der Daten wird das Entity-Framework verwendet mit welchem bereits in einem früheren Projekt Erfahrung gesammelt werden konnte. 

\section{Spezifikation der Systemanforderungen}
%Dieses Kapitel beschreibt die funktionalen und nichtfunktionalen Anforderungen genauer. Gegebenenfalls können auch weitere Einzelheiten zu den nichtfunktionalen Anforderungen hinzugefügt werden. Schnittstellen zu anderen Systemen können definiert werden.
\section{Systemmodelle}
%Hier sind grafische Systemmodelle enthalten, um die Beziehungen zwischen den Systemkomponenten und dem System und seiner Umgebung aufzuzeigen. Beispiele für Modelle sind Objekt-, Datenfluss- und semantische Datenmodelle
\section{Systemevolution}
%Dieser Teil beschreibt die grundlegend en Voraussetzungen, auf denen das System basiert, und jede erwartete Veränderung aufgrund der Hardwareentwicklung, Veränderungen der Benutzeranforderungen usw. Dieser Abschnitt ist für Systementwickler nützlich, da er helfen  kann, Entwurfsentscheidungen zu vermeiden, die voraussichtliche Änderungen in der Zukunft am System beschränken würde
